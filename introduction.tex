\chapter*{Введение}
\addcontentsline{toc}{chapter}{Введение}

На современном этапе развития человечества довольно сложно представить мир без искусственного интеллекта, который продолжает развиваться и захватывает всё новые сферы и области человеческой деятельности. Сейчас ИИ способен обеспечить общение между устройством и человеком, благодаря его способности распознавать голос и фразы человека; распознавая автомобили, способен дать ценные данные о загруженности улиц в часы пик; а наделённые высокими мощностями ИИ сейчас борется за жизнь и здоровье человека, выполняя одну из самых сложнейших задач: выставление диагнозов и определение способов лечения той или иной болезни с учётом индивидуальных особенностей пациента. 

Но самой ценной чертой ИИ, которая нас интересует, является его желание заглянуть в будущее. Как и человек, ИИ тоже хочет знать, что будет происходить в будущем, однако в отличии от человека, у ИИ есть возможности это сделать и при этом претендовать на достоверность своего видения. Самая очевидная подзадача предвидения будущего — это прогнозирование, например, работы некоторой установки, последствий принятия решений, тенденций развития того или иного объекта. Нас, по большей части, интересует прогнозирование работы устройств, их технологического процесса, поскольку это один из ключевых моментов производства, позволяющий повысить эффективность управления им. Ведь в современных условиях управления производством, которые, к слову, становятся всё сложнее, а требования к эффективности взлетают всё выше, выигрышным вариантом становятся системы нейроуправления, одной из задач которых является и прогнозирование. Ключевое качество таких систем заключается в обучение во время функционирования систем, что позволяет повысить эффективность таких систем и достигать более лучших результатов.

Решение задачи прогнозирования именно нейросетевым способом на сегодняшний день является одним из самых актуальных, эффективных и точных способов решения данной задачи. Мало того, что такой способ позволяет гибко адаптироваться к процессам и научиться достаточно точно их прогнозировать, так ещё и такой способ предусматривает возможность, так скажем, дообучения, благодаря которому, можно улавливать новые особенности процесса. А если появление новых особенностей в процессе не предусмотрено, тогда у нас появляются основания считать, что система смогла засечь выбросы или аномалии в процессе. Благодаря этому можно будет принять необходимое решение, чтобы восстановить или как-то повлиять на работу устройства, где в процессе прогнозирования была выявлена аномалия. 

Но что мы будем прогнозировать? Как прогнозировать технологический процесс? Чтобы нам это сделать, нам необходимо работать с временными рядами, поскольку они – это один из способов описания технологического процесса, который может хранить информацию о состоянии процесса в определённые моменты времени, что нам и нужно. Таким образом, прогнозирование данных временных рядов позволяет прогнозировать сам технологический процесс. Что и является основной целью данной работы.
