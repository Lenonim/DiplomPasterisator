\sectionbreak \section*{
	\gostTitleFont
	\redline
	ЗАКЛЮЧЕНИЕ
}

\subtitlespace

{\gostFont
	\par \redline В рамках данного дипломного проекта было разработано программное обеспечения для прогнозирования данных временных рядов технологического процесса пастеризационной установки с учётом зависимости параметров процесса между собой. 
	
    \par \redline В процессе выполнения дипломного проекта были решены следующие задачи:
	
	\begin{itemize}[leftmargin=2.15cm, labelwidth=0.65cm, labelsep=0.0cm] 
		
		\item[\theitemcntr.] Проектирование системы.		
		\addtocounter{itemcntr}{1}
		
		\item[\theitemcntr.] Выбор архитектуры нейронной сети.		
		\addtocounter{itemcntr}{1}
		
		\item[\theitemcntr.] Настройка системы и средств реализации, получение системы реализации. 		
		\addtocounter{itemcntr}{1}
		
		\item[\theitemcntr.] Обеспечение для системы возможность адаптации под нужны конкретного производства, работать в нескольких режимах.		
		\addtocounter{itemcntr}{1}
		
		\item[\theitemcntr.] Адаптация способов обучения под конкретную архитектуру нейронной сети.		
		\addtocounter{itemcntr}{1}
		
		\item[\theitemcntr.] Тестирование на корректность исполнения отдельных частей.		
		\addtocounter{itemcntr}{1}
		
		\item[\theitemcntr.] Реализовано непосредственное прогнозирование данных технологического процесса.		
		\addtocounter{itemcntr}{1}
		
		\item[\theitemcntr.] Разработка средства подготовки данных.		
		\addtocounter{itemcntr}{1}
		
		\item[\theitemcntr.] Реализация сохранения конфигурации модели нейронной сети.		
		\addtocounter{itemcntr}{1}
		
		\item[\theitemcntr.] Реализация возможности дополнительного обучения и редактирования модели нейронной сети.		
		\addtocounter{itemcntr}{1}

		\setcounter{itemcntr}{1}
	\end{itemize} 

	\par \redline Данное программное обеспечение удовлетворяет всем поставленным требованиям, а именно:
	
	\begin{itemize}[leftmargin=2.15cm, labelwidth=0.65cm, labelsep=0.0cm] 
		
		\item[\theitemcntr.] Возможность формирование прогнозов.	
		\addtocounter{itemcntr}{1}
		
		\item[\theitemcntr.] Интеграция на контроллеры PLCNext.		
		\addtocounter{itemcntr}{1}
		
		\item[\theitemcntr.] Интеграция с внешними системами. 		
		\addtocounter{itemcntr}{1}
		
		\item[\theitemcntr.] Программа устойчива к сбоям.		
		\addtocounter{itemcntr}{1}
		
		\item[\theitemcntr.] Программа не генерирует сбои.		
		\addtocounter{itemcntr}{1}
		
		\setcounter{itemcntr}{1}
	\end{itemize}  
	  
	\par \redline Проведенное тестирование показало ожидаемый результат, что говорит о правильной реализации функционала программного обеспечения. 

	\par \redline Для разработанного программного обеспечения был осуществлен расчёт экономической эффективности, а также определена стоимость проекта и прибыль разработчика от реализации проекта.
	
}

\setcounter{subchaptercntr}{1}
\setcounter{formulacntr}{1}
\setcounter{imagecntr}{1}