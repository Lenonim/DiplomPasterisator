\sectionbreak \section*{ 
    \cyrillicfont
    \fontsize{14pt}{24pt}\selectfont 
    \redline
    ВВЕДЕНИЕ
}
\titlespace

{\cyrillicfont
\fontsize{13pt}{16.25pt}\selectfont 
\englishfont 

\par \redline На современном этапе развития человечества довольно сложно представить мир без нейронных сетей, которые продолжают развиваться и осваивать всё новые сферы и области человеческой деятельности. Одной из таких хорошо освоенных нейронными сетями сфер является производственная сфера. Обученные нейронные сети способны классифицировать продукты производства и определять их дальнейшую судьбу, способны предсказывать поведение некоторой системы для дальнейшего планирования производства, искать аномалии в данных технологического процесса и многое другое. 

\par \redline Данная работа как раз и посвящена прогнозированию данных технологического процесса производственного оборудования, а именно пастеризационной установки. Для выполнения данной задачи будут использоваться нейронные сети, поскольку решение задачи прогнозирования именно нейросетевым способом на сегодняшний день является одним из самых актуальных, эффективных, точных, изученных и разнообразных способов решения данной задачи. Мало того, что такой способ позволяет гибко адаптироваться к процессам и научиться достаточно точно их прогнозировать, так ещё и такой способ предусматривает возможность, так скажем, подстраивания, благодаря чему, можно улавливать новые особенности процесса. 

\par \redline По большей части, данная работа является смесью задачи прогнозирования и машинного обучения, поскольку в её рамках будет рассматриваться изучение предметной области и определение задач и целей прогнозирования, сбор, анализ и обработка данных технологического процесса пастеризационной установки, их подготовка к прогнозирования и обучнию нейронной сети, после проектирование и реализация самой нейронной сети, её обучение, тестирование и оценка, а в заключительном этапе готовую модель необходимо развернуть для взаимодействия с пастеризационной установкой.

\par \redline Дойдя до этапа создания модели, будут выбраны основные архитектуры нейронных сетей, подробно описаны их способы функционирования и обучения, а также детально будут рассмотрены результаты работы моделей и их сравнение между собой.

\par
}
