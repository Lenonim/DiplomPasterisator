\sectionbreak \section*{АНАЛИЗ ПРЕДМЕТНОЙ ОБЛАСТИ И ПОСТАНОВКА ЗАДАЧИ}
\addcontentsline{toc}{chapter}{АНАЛИЗ ПРЕДМЕТНОЙ ОБЛАСТИ И ПОСТАНОВКА ЗАДАЧИ}

\section*{Основные понятия предметной области}

\section*{Результаты обследования пастеризационной установки}

\section*{Введение в прогнозирвоание и анализ временных рядов}

{\cyrillicfont % используем TNR для символов киррилицы
\fontsize{13pt}{16.25pt}\selectfont % настраиваем размер шрифта
\setlength{\parindent}{1.25cm} % настраиваем красную строку
\englishfont

\par Одно из ключевых понятий данной работы является понятие временных рядов. Что же это? Временные ряды – это, по сути, некоторая последовательность, каждый элемент из которой состоит из двух или более признаков, а один из которых обязательно должен обозначать время. Причём, все эти элементы в последовательности расположены в хронологическом порядке, т.е. в порядке возрастания времени. 

\par Признак времени может быть представлен в разных форматах. Выбор формата времени зависит от задачи, удобства использования, длительности, в пределах которой будут собираться данные, а также от требуемой точности. Например, в случае, если данные фиксируются раз в день, то хорошо подойдёт отсчёт времени по дням с указанием месяца и года. Если же данные фиксируются в определённые моменты дня, то к вышеописанному стоит прибавить указание часа и минуты фиксации. При необходимости можно указывать и секунды, и миллисекунды. Но что, если нам не особо-то и важно знать, в какой год, месяц или день это происходило, когда нам важно знать, сколько прошло времени от начала того или иного процесса? Тогда, нам скорее подойдёт формат дискретного времени. С помощью этого формата мы может узнать длительность процесса в единственной выбранной нами единице времени. Например, если мы сохраняем время в секундах, то 1000 секунд и сохранит свой формат 1000 секунд, время не будет переведено в 16 минут и 40 секунд. Всё это нам позволяет не привязываться к датам, которые не особо-то и влияют на технологический процесс пастеризационной установки.

\par Остальные признаки могут уже характеризовать, описывать какой-либо процесс или процессы, причём даже не обязательно одного элемента, а целой системы элементов. Так, например, когда наш элемент последовательности состоит из двух признаков, а один из которых, как мы уже знаем, время, то второй, конечно же, уже будет обозначать характеристику или состояние изучаемого нами элемента. Но как только у нас появляется три или более признаков, тогда мы можем говорить о фиксации характеристик или состояний разных элементов изучаемой системы в один и тот же момент времени или же о фиксации характеристики или состояния одного из множества элементов системы, но с указание этого элемента, например, с помощью идентификационного номера элемента. Ну и так далее. Поэтому временные ряды даёт весьма гибкую возможность описания процессов или систем относительно времени. 

\par А что из себя представляет работа с временными рядами? В основном работа делится на две части. Первая часть – это понимание структуры временного ряда, его закономерностей, таких как цикличность, тренд, сезонность и так далее, обработка данных временного ряда, в общем, это анализ временного ряда. Если опускать различные математические, статистические и тому подобные подробности, то анализ также может дать нам возможность понять, как начинался процесс, как шёл и развивался и на чём он закончился или остановился на данном моменте времени.  Строго говоря, нейронным сетям, как и исследователям временного ряда, тоже необходимо это понять, чтобы выполнить вторую часть работы, а именно, составление прогноза, что, зачастую, и является основной задачей работы с временным рядом. Да, анализ данных временного ряда делался с целью понять, что будет происходить с процессом дальше. Чем нам так полезна информация о будущем? Для этого необходимо понять, что есть прогноз. 

\par
}