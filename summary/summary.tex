{\gostTitleFont
    \redline
    СОДЕРЖАНИЕ
} 

\titlespace

{\gostFont

    \par {\bfseries ВВЕДЕНИЕ \dotfill 5}
    \par {\bfseries 1. АНАЛИЗ ПРЕДМЕТНОЙ ОБЛАСТИ И ПОСТАНОВКА ЗАДАЧИ \dotfill 6}
    \par 1.1 Основные понятия предметной области \dotfill 6
    \par 1.2 Результаты обследования пастеризационной установки \dotfill 7 
    \par 1.3 Введение в прогнозирвоание, нейронные сети и машинное обучение \dotfill 10 
    \par 1.4 Требования к модулю прогнозирования и постановка задачи \dotfill 15 
    \par {\bfseries 2. АНАЛИЗ СУЩЕСТВУЮЩИХ РЕШЕНИЙ \dotfill 16}
    \par {\bfseries 3. АНАЛИЗ И ОБРАБОТКА ДАННЫХ ПАСТЕРИЗАЦИОННОЙ УСТАНОВКИ \dotfill 20}
    \par 3.1 Анализ и визуализация исходных данных технологического прцоесса \dotfill 20
    \par 3.2 Обработка данныз и их подготовка к прогнозированию \dotfill 29
    \par {\bfseries 4. РАЗРАБОТКА, ОБУЧЕНИЕ И ОЦЕНКА МОДЕЛИ НЕЙРОННОЙ \dotfill 30}
    \par 4.1 Выбор архитектуры нейронной сети для прогнозирования \dotfill 30
    \par 4.2 Обзор LSTM как средства прогнозирования \dotfill 31
    \par 4.3 Обучение сети LSTM \dotfill 35
    \par 4.4 Гиперпараметры сети LSTM \dotfill 39
    \par 4.5 Разработка модуля прогнозирования \dotfill 40
    \par 4.6 Тестирование и оценка модели нейронной сети \dotfill 41
    \par {\bfseries 5. РАЗВЁРТЫВАНИЕ МОДЕЛИ НЕЙРОННОЙ СЕТИ \dotfill 47}
    \par 5.1 Основные виды развёртывания \dotfill 47
    \par 5.2 Технология PLCnext и контроллер AXC F 2152 \dotfill 48
    \par 5.3 Выбор средств сборки под контроллер на платформе PLCnext \dotfill 51
    \par 5.4 Выбор средств установки программ на платформу PLCnext \dotfill 52
    \par 5.5 Портирование программы на контроллере \dotfill 54
    \par {\bfseries 6. ТЕХНИКО-ЭКОНОМИЧЕСКОЕ ОБОСНОВАНИЕ \dotfill 55}
    \par 6.1 Исходные данные для расчета экономического эффекта \dotfill 55
    \par 6.2 Расчёт объёма функций программного модуля \dotfill 55
    \par 6.3 Расчёт полной себестоимости программного модуля \dotfill 56
    \par 6.4 Расчет цены и прибыли по программному продукту \dotfill 60
    \par {\bfseries ЗАКЛЮЧЕНИЕ \dotfill 63} 
    \par {\bfseries СПИСОК СОКРАЩЕНИЙ \dotfill 64} 
    \par {\bfseries СПИСОК ЛИТЕРАТУРЫ \dotfill 65} 
    \par {\bfseries ПРИЛОЖЕНИЕ А: ТЕКСТ ПРОГРАММЫ} 
    \par 
}