\documentclass[a4paper, oneside, openany]{report}
% a4paper - формат листа А4
% oneside - односторонний вывод
% openany - начинаем новую главу со следующей новой страницей
% 14pt - стандартный размер шрифта 

\usepackage{ragged2e} % патек для нормального выравнивания текста

\usepackage{geometry} % пакет для работы с разметкой страницы
\geometry{left=2.6cm} % левый отступ
\geometry{right=1.1cm} % правый отступ
\geometry{top=2.22cm} % верхний отступ
\geometry{bottom=2.6cm} % нижний отступ

\usepackage{titlesec} % пакет для работы с разделами 
\newcommand{\sectionbreak}{\clearpage} % что каждый раздел начинался с новой страницы

\usepackage{parskip} % сам настроит интервалы нужным образом

\setcounter{page}{4} % устанавливаем начальную страницу

% настройка шрифтов и языков
    \usepackage[english, russian]{babel} % для работы с английским и русским языками
    \usepackage[T2A]{fontenc} % для норм паказа символов в pdf
    \usepackage[utf8]{inputenc} % для норм взаимодействия символов для самого tex

    \usepackage{fontspec} % для работы со шрифтами 
    \usepackage{polyglossia} % Поддержка многоязычности
    \defaultfontfeatures{Ligatures=TeX,Mapping=tex-text}

    \setmainlanguage[babelshorthands = true]{russian} % устанавливаемя русский как основной язык
    \setotherlanguage{english} % устанавливаем английский как дополнительный язык

    \setmainfont{Times New Roman} % устанавливаем главный шрифт 
    \setsansfont[Ligatures=TeX]{Arial} % устонавливаем дополнительный шрмфт 

    \newfontfamily\cyrillicfont[Script=Cyrillic]{Times New Roman} % делаем так, чтобы русский язык мог отображаться главным шрифтом 
    \newfontfamily\cyrillicfontsf[Script=Cyrillic]{Arial} % делаем так, чтобы русский язык мог отображаться дополнительным шрифтом 

    \newfontfamily\englishfont{Times New Roman} % делаем так, чтобы английский язык мог отображаться главным шрифтом 
    \newfontfamily\englishfontsf{Arial} % делаем так, чтобы английский язык мог отображаться дополнительным шрифтом 

% Определяем переменные 
    \newcommand\redline{\hspace{1.5cm}} % красная строка
    \newcommand\titlespace{\vspace{20pt}} % промежуток между заголовками 
    \newcommand\subtitlespace{\vspace{20pt}} % промежуток между заголовком и текстом 
    \newcommand\gostFont{\cyrillicfont \englishfont \fontsize{13pt}{15.6pt}\selectfont} % набор необходимых параметров для обычного текста
    \newcommand\gostTitleFont{\cyrillicfont \englishfont \fontsize{14pt}{16.8pt}\selectfont \bfseries} % набор необходимых параметров для текста заголовков
    \newcommand\gostRamkaFont{\itshape \cyrillicfontsf \fontsize{8pt}{0pt}\selectfont}
    \newcommand\bigGostRamkaFont{\itshape \cyrillicfontsf \fontsize{8pt}{0pt}\selectfont}
    \parskip=0pt % чтобы между абзацами было расстояние равное межстрочному интервалу


% Рисует рамку
    \usepackage{xltxtra}

    \textheight=260mm
    \textwidth=175mm
    \unitlength=1mm
    
    \oddsidemargin=-0.5mm
    \topmargin=-2.45cm

    \def\VL{\line(0,1){15}}
    \def\Box#1#2{\makebox(#1,5){#2}}
    \def\simpleGrad{\noindent\hbox to 0pt{
            \vbox to 0pt{
                \noindent\begin{picture}(185,286.75)(5, 0)
                    \linethickness{0.5mm}
                    \put(-0.5, 0){\framebox(184.5,287){}}
                    \put(-0.5, 25){\Box{7}{\gostRamkaFont Изм.}}
                    \put(-0.5, 40){\line(1,0){185}}

                    \linethickness{0.25mm}
                    \put(-0.5, 30){\line(1,0){65}}
                    \put(-0.5, 35){\line(1,0){65}}
                    \linethickness{0.5mm}

                    \put(6.5, 25){\line(0,1){15}}
                    \put(6.5, 25){\Box{10}{\gostRamkaFont Лист.}}
                    \put(16.5, 0){\line(0,1){40}}
                    \put(16.5, 24.5){\Box{15}{\gostRamkaFont № докум.}}
                    \put(39.5, 0){\line(0,1){40}}
                    \put(39.5, 25){\Box{10}{\gostRamkaFont Подп.}}
                    \put(54.5, 0){\line(0,1){40}}
                    \put(54.5, 25){\Box{10}{\gostRamkaFont Дата}}
                    \put(64.5, 0){\line(0,1){40}}
                    \put(64.5, 25){\makebox(120,15){\cyrillicfont \bfseries \englishfont \fontsize{18pt}{0pt}\selectfont ДП.АС59.200043-05 81 00}}
                    \put(-0.5, 25){\line(1,0){185}}

                    \linethickness{0.25mm}
                    \put(-0.5, 5){\line(1,0){65}}
                    \put(-0.5, 10){\line(1,0){65}}
                    \put(-0.5, 15){\line(1,0){65}}
                    \put(-0.5, 20){\line(1,0){65}}
                    \linethickness{0.5mm}

                    \put(-0.5, 20){\Box{12}{\gostRamkaFont Разраб. }}
                    \put(-0.5, 15){\Box{14.5}{\gostRamkaFont Проверил }}
                    \put(-0.5, 5){\Box{13.5}{\gostRamkaFont Н контр. }}
                    \put(-0.5, 0){\Box{7.1}{\gostRamkaFont  Утв.}}

                    \put(16.5, 20){\Box{20}{\gostRamkaFont Абоимов И.В. }}
                    \put(16.5, 15){\Box{18.25}{\gostRamkaFont Иванюк Д.С. }}
                    \put(16.5, 5){\Box{16}{\gostRamkaFont Булей Е.В. }}
                    \put(16.5, 0){\Box{19}{\gostRamkaFont  Головко В.А. }}

                    \put(134.5, 0){\line(0,1){25}}
                    \put(139.5, 15){\line(0,1){5}}
                    \put(144.5, 15){\line(0,1){5}}
                    \put(134.5, 20){\line(1,0){50}}
                    \put(134.5, 15){\line(1,0){50}}
                    \put(149.5, 15){\line(0,1){10}}
                    \put(164.5, 15){\line(0,1){10}}

                    \put(64.5, 0){\makebox(70,40){\cyrillicfont \englishfont \fontsize{9pt}{0pt}\selectfont \itshape Прогнозирование данных }}
                    \put(64.5, 0){\makebox(70,30){\cyrillicfont \englishfont \fontsize{9pt}{0pt}\selectfont \itshape временных рядов технологического процесса}}
                    \put(64.5, 0){\makebox(70,20){\cyrillicfont \englishfont \fontsize{9pt}{0pt}\selectfont \itshape пастеризационной уставноки}}
                    \put(64.5, 0){\makebox(70,7){\cyrillicfont \englishfont \fontsize{9pt}{0pt}\selectfont \itshape Пояснительная записка}}
                    \put(134.5, 0){\makebox(49,15){\cyrillicfont \englishfont \fontsize{16pt}{0pt}\selectfont \itshape УО БрГТУ}}
                    \put(134.5, 20){\Box{15}{\gostRamkaFont Лит }}
                    \put(134.5, 15){\Box{5}{\gostRamkaFont Д }}
                    \put(149.5, 20){\Box{15}{\gostRamkaFont Лист }}
                    \put(149.5, 15){\Box{15}{\gostRamkaFont \thepage }}
                    \put(164.5, 20){\Box{19}{\gostRamkaFont Листов }}
                    \put(164.5, 15){\Box{19}{\gostRamkaFont 66 }}

                \end{picture}
            }
        }
    }
    
    \makeatletter
    \def\@oddhead{\simpleGrad}
    \def\@oddfoot{}
    \makeatother

    \newcounter{chaptercntr}
    \setcounter{chaptercntr}{1}
    \newcounter{subchaptercntr}
    \setcounter{subchaptercntr}{1}

    \newcommand\spc{\hspace{0.08cm}} 
