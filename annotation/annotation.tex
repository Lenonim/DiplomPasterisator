\documentclass[a4paper, oneside]{report}

\usepackage{ragged2e} 

\usepackage{geometry}
\geometry{left=3cm}
\geometry{right=1.5cm}
\geometry{top=3.0cm} 
\geometry{bottom=2.0cm}

\usepackage{titlesec} 
\newcommand{\sectionbreak}{\clearpage} 

\usepackage{graphicx, epsfig}
\usepackage[inkscapeformat=png]{svg} 

\usepackage{parskip} % сам настроит интервалы нужным образом

\setcounter{page}{5} % устанавливаем начальную страницу

\usepackage{siunitx}

% настройка шрифтов и языков
\usepackage[english, russian]{babel} % для работы с английским и русским языками
\usepackage[T2A]{fontenc} % для норм паказа символов в pdf
\usepackage[utf8]{inputenc} % для норм взаимодействия символов для самого tex

\usepackage{fontspec} % для работы со шрифтами 
\usepackage{polyglossia} % Поддержка многоязычности
\defaultfontfeatures{Ligatures=TeX,Mapping=tex-text}

\setmainlanguage[babelshorthands = true]{russian} % устанавливаемя русский как основной язык
\setotherlanguage{english} % устанавливаем английский как дополнительный язык

\setmainfont{Times New Roman} % устанавливаем главный шрифт 
\setsansfont[Ligatures=TeX]{Arial} % устонавливаем дополнительный шрмфт 

\newfontfamily\cyrillicfont[Script=Cyrillic]{Times New Roman} % делаем так, чтобы русский язык мог отображаться главным шрифтом 
\newfontfamily\cyrillicfontsf[Script=Cyrillic]{Arial} % делаем так, чтобы русский язык мог отображаться дополнительным шрифтом 

\newfontfamily\englishfont{Times New Roman} % делаем так, чтобы английский язык мог отображаться главным шрифтом 
\newfontfamily\englishfontsf{Arial} % делаем так, чтобы английский язык мог отображаться дополнительным шрифтом 

% Определяем переменные
\newcommand\spc{\hspace{0.08cm}} % отступ от каунтера
\newcommand\redline{\hspace{1.5cm}} % красная строка
\newcommand\formulaspace{\vspace{0.4cm}} % красная строка
\newcommand\topTablespace{\vspace{15.6pt}} % отступ до начала таблиц
\newcommand\botTablespace{\vspace{18.2pt}} % отступ от конца таблиц
\newcommand\wherespace{\hspace{0.7cm}} % отступ для пояслений к формулам
\newcommand\titlespace{\vspace{26pt}} % промежуток между заголовками 
\newcommand\subtitlespace{\vspace{39pt}} % промежуток между заголовком и текстом 
\newcommand\gostFont{\cyrillicfont \englishfont \fontsize{13pt}{15.6pt}\selectfont} % набор необходимых параметров для обычного текста
\newcommand\gostTitleFont{\cyrillicfont \englishfont \fontsize{13pt}{0pt}\selectfont \bfseries} % набор необходимых параметров для текста заголовков
\newcommand\gostRamkaFont{\itshape \cyrillicfontsf \fontsize{8pt}{0pt}\selectfont}
\parskip=0pt % чтобы между абзацами было расстояние равное межстрочному интервалу

\newcommand\FNine{\fontsize{9pt}{10.8pt}\selectfont} 
\newcommand\FTwelwe{\fontsize{12pt}{14.4pt}\selectfont}

\usepackage{fancyhdr}

\textheight=260mm
\textwidth=175mm
\unitlength=1mm

\oddsidemargin=-0.5mm
\topmargin=-2.45cm

\begin{document} 

	%\pagestyle{fancy}

	%\fancyhf{} % sets both header and footer to nothing
	%\renewcommand{\headrulewidth}{0pt}

	%\fancyhead[C]{In the centre of the header on all pages: \thepage}
	%\fancyfoot[RO]{\thepage}
	

	\sectionbreak \section*{
		\gostTitleFont
		\redline 
		АННОТАЦИЯ
	}
	
	\titlespace
	
	{\gostFont
	
	\par \redline 66 с./ 73 с.,  24 рис.,  4 табл., 14 исп. ист., 1 прил., 5 граф. матер.
	
	\par \redline 

	\par \redline Целью данного дипломного проекта является создание программного обеспечения для прогнозирования данных временных рядов технологического процесса пастеризационной установки. 
	
	\par \redline Приложение разработано с использованием среды разработки Visual Studio 2022 и предназначено для работы под управлением контроллера на основе технологии PLCNext Technology. 
	
	\par \redline Разработанное приложение обеспечивает возможность создания, сохронения, загрузки моделей нейронной сети, обучения и редактирования модели нейронной сети, непосредственного прогнозирования данных временных рядов технологического процесса пастеризационной установки.
	 
	\par \redline Проект включает в себя следующие разделы: введение; анализ предметной области и постановка задачи; анализ существующих решений; анализ и обработка данных пастеризационной установки; разработка, обучение и оценка модели нейронной сети; развертывание модели нейронной сети; технико-экономическое обоснование; заключение. 

	\par 
	
    }
	
\end{document}
