\sectionbreak \section*{ 
    \gostTitleFont
    \redline
    Выполненные индивидуальной задачи
}

\subtitlespace

{\gostFont

    \par \redline В ходе практики было уделено время и задачам ОАО <<Савушкин Продукт>>, цель которых состоит в актуализации проекта SCADA-системы. 

    \par \redline Прежде, чем перейти к описанию результатов выполненный задач, немного предыстории и некоторых пояснений. SCADA-система это огромный проект необходимый для контроля и управления производством. В этой системе можно создать модель любого технологического оборудования и эмулировать его работу, его технологический процесс. Данный проект берёт своё начало аж в 90-ые года, но и по сей день эта система используется на ОАО <<Савушкин Продукт>> и успешно справляется со своими задачами. Однако технологический процесс не стоит на месте: появляется новое компьютерное оснащение, операционные системы, новые версии IDE, стандарты и компоненты языка программирования и так далее. Задача актуализации проекта SCADA-системы заключается как раз именно в том, чтобы адаптировать эту систему под новейшие технологии и средства, исправив ошибки совместимости, а также заключается и в устранении сторонних компонентов, которые были использованы по тем или иным причинам при разработки системы и которые необходимо заменить на стандартные средства языка программирования.

    \par \redline Ранее мне была поставлена задача, которая заключается в удалении RX компонентов из SCADA-системы. Большая часть таких компонентов была заменена. В ходе практики было уделено время на устранение оставшихся RX-компонентов, а также написание тестов для заменённых на стандартные версии компонентов.

    \par \redline На данном этапе актуализации проекта SCADA-системы произошло знакомство с DUtins, они же unit-тесты для delphi. В ходе создания проектов тестов пришлось столкнуться с рядом проблем, на решение которых ушло время, но это стало ценным опытом.

    \par \redline Во-первых, при создании проекта тестов требуется указать все те же связи, что есть и в проекте, для которого создаются тесты, а их могут быть десятки, что забирает некоторое время на настройку проекта теста. 

    \par \redline Во-вторых, можно столкнуться с дерективами препроцессора, которые перманенто могут вызывать ошибки компиляции проекта тестов. На первый взгляд, можно определить неоходимые переменные, чтобы проходить или не проходить в определённые участки кода, но где это делать? А если таких переменных тоже десятки? Решение этой проблемы можно произвести относительно быстро: настроить Debuger для проекта тестов в соответствии с настройками Debuger тестируемого проекта. Это можно сделать следующим образом (пример на IDE Embarcadero Delphi XE5): Project $\rightarrow$ Options $\rightarrow$ Delphi Compiler $\rightarrow$ Conditional defines. Далее прописать там все определённые в тестируемом проекте переменные. 

    \par \redline В-третьих, некоторые компоненты, например, TeeEngine, отказываются работать и выдают ошибку подключения своих библиотек, несмотря на то, что они числятся в компонентах проекта как подключённые или активные. Для решения данной проблемы необходимо настроить сам проект тестов в соответствии с тестируемом проектом. Это можно сделать следующим образом (пример на IDE Embarcadero Delphi XE5): Project $\rightarrow$ Options $\rightarrow$ Delphi Compiler $\rightarrow$ Unit score names. Далее прописать там все определённые в тестируемом проекте зависимости. Если это не решило проекта, тогда необходимо прописать зависимости в Project $\rightarrow$ Options $\rightarrow$ Packages $\rightarrow$ Runtime Packages $\rightarrow$ Runtime Packages.

    \par \redline В-четвёртых, происходит автоматическая генерация кода с ошибками, которые также необходимо устранять. В основном ошибки генерации кода встречаются в файлах формата .dpr, т.е. в файлах проекта. 

    \par \redline В-пятых, после всех настроект конфигурации проекта, всё равно будут различия между проектами, которые кроются в деталях. Одной из таких деталей является различие десятичного сепаратора. Его тоже необходимо настроить вручную, использовав переменную FormatSettings.DecimalSeparator. 

    \par \redline Преодолев все вышеописанные трудности, появилась возможность написания самих тестов. В ходе этого был разработан ряд тестов, проверяющих как заменённые визуальные компоненты, так и различные системыные функции.  

    \par \redline Для тестирования системных функций были созданы простейшие проекты средствами delphi. Это было сделано по причине того, что системные функции, которые необходимо проверить, являются частью других огромных функций или процедур, а протестировать необходимо лишь пару строчек. 

    \par \redline Также мною была продолжена работа по замене сторонних компонентов на системные компоненты, которые предоставляет язык программирования Delphi. 

    \par \redline Удаление RX компонентов из проекта DBEditorXML является непростой задачей. В этом проекте присутствует специфический компонент, аналогов которого стандартный Delphi предложить не может. Данный компонент использует ini-файлы и предназначен для сохранения и загрузки конфигурации окна. 

    \par \redline Для замены этого компонента был проведён анализ ini-файла для того, чтобы заменить специфический компонент, средствами объекта TIniFile.   

    \par \redline Были выявлены места, где производится автоматическая запись данных в ini-файл и где производится автоматическое считывание данных из ini-файла. Для этого один из сотрудников предоставил предзаполненную тестовую базу данных, которая открывается и редактируется средствами DBEdtiorXML. Для замены специфического компонента были созданы функции, позволяющие сохранять конфигурацию окна. 

    \par
}