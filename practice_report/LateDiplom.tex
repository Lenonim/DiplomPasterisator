\sectionbreak \section*{
  \gostTitleFont
  \redline
  Дальнейшая структура диплома
}

\subtitlespace

{\gostFont

  \par \redline Дальнейшая структура диплома содержит в себе ещё несколько разделов. Вкратце разберём каждый из них.

  \par \redline {\bfseries 3. АНАЛИЗ И ОБРАБОТКА ДАННЫХ ПАСТЕРИЗАЦИОННОЙ УСТАНОВКИ}

  \par \redline В данном разделе планируется подробно рассмотреть исходные данные временных рядов технологического процесса пастеризационной установки. Под этим понимается: анализ и определение форматов хранения как исходных, так и обработанных данных; способы представления данных технологического процесса; визуализация данных и инфографика, расскрывающая особенности технологического процесса; все способы обработки данных и их математические основы. Но самое важное, что должно быть в этом разделе, так это определение взаимосвязей данных различных сидов для составления наиболее достоверных прогнозов. 

  \par \redline {\bfseries 4. РАЗРАБОТКА И ОБУЧЕНИЕ МОДЕЛИ НЕЙРОННОЙ СЕТИ}

  \par \redline В этом разделе будет обоснован выбор именно реккурентных нейронных сетей, а в частности LSTM сетей для прогнозирования временных рядов. Будут рассмотрены особенности создания и обучения модели нейронной сети, а также все особенности разработки, такие как: какая операционная система, IDE, платформа, язык программирования. Поскольку всё реализовано с помощью стандартных средств C++ то описание сторонних библиотек будет отсутствовать. Также будут разобраны такие моменты, почему существует три модуля, как они взаимосвязаны между собой и зачем каждый из них нужен. 

  \par \redline {\bfseries 5. ТЕСТИРОВАНИЕ И ОЦЕНИВАНИЕ МОДЕЛИ ПРОГНОЗИРОВАНИЯ}

  \par \redline Как понятно из названия раздела, здесь будут описаны все средства, которые использовались для тестирования, настройки и оценивания модели нейронной сети.

  \par \redline {\bfseries 6. РАЗВЁРТЫВАНИЕ МОДЕЛИ ПРОГНОЗИРОВАНИЯ}

  \par \redline В данном разделе будет рассмотрен программируемый микроконтроллер, на который необходимо будет перенести обученную модель прогнозирования. Также будет рассмотрено место этого контроллера в самой пастеризационной установке для взаимодействия с ней. Будут показаны особенности подачи данных на микроконтроллер и получение прогноза с этого контроллера.  

  \par \redline Далее идут разделы посвящённые охране труда и техникоэкономическому обоснованию, затем заключение, список сокращений и литературы. После идут приложения с текстом программы модуля прогнозирования и инстукцией по применению. 

  \par
}