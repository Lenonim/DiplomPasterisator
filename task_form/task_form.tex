\documentclass[12pt, А4, twoside]{article} % задаём тип документа: статья - и его характеристики: 12-ый стандартный шрифт, размер листа А4

    % Внимание! Чтобы код ниже работал, необхоидмо переключиться на движок XeLaTe
    \usepackage{fontspec} % подключаем модуль для работы со шрифтами
    \PassOptionsToPackage{no-math}{fontspec} %  https://tex.stackexchange.com/a/26295/104425
    \usepackage{polyglossia} % Поддержка многоязычности
    \usepackage{ragged2e} % патек для нормального выравнивания текста
    \usepackage[ % задаём макет документа
        left=2.8cm, % отступ от левого края 
        right=1.3cm, % отступ от правого края
        top=2.0cm, % отсутп сверху
        bottom=1.59cm, % отступ снизу
    ]{geometry} 

    % настройка шрифтов и языков
    \defaultfontfeatures{Ligatures=TeX,Mapping=tex-text}

    \setmainlanguage[babelshorthands = true]{russian} % устанавливаемя русский как основной язык
    \setotherlanguage{english} % устанавливаем английский как дополнительный язык

    \defaultfontfeatures[ArrialNarrow]{ % подключаем шрифт из файлов
        Path = fonts/,
        Extension = .ttf,
        UprightFont = arialnarrow,
        BoldFont = arialnarrow_bold,
        ItalicFont = arialnarrow_italic,
        BoldItalicFont = arialnarrow_bolditalic,
    }

    \setmainfont{Times New Roman} % устанавливаем главный шрифт 
    \setsansfont[Ligatures=TeX]{ArialNarrow} % устонавливаем дополнительный шрмфт 

    \newfontfamily\cyrillicfont[Script=Cyrillic]{Times New Roman} % делаем так, чтобы русский язык мог отображаться главным шрифтом 
    \newfontfamily\cyrillicfontsf[Script=Cyrillic]{ArialNarrow} % делаем так, чтобы русский язык мог отображаться дополнительным шрифтом 

    \newfontfamily\englishfont{Times New Roman} % делаем так, чтобы английский язык мог отображаться главным шрифтом 
    \newfontfamily\englishfontsf{ArialNarrow} % делаем так, чтобы английский язык мог отображаться дополнительным шрифтом 

    \pagenumbering{gobble} % убирает нумерацию

\begin{document} % начала тела документа, его основной части

\begin{FlushRight}{\fontsize{11}{13.75}\selectfont \cyrillicfont
        Форма №17
} \end{FlushRight}
    % создали область видиния { <здесь текст, который подчиняется заданым настройкам> }
    % \fontsize{x}{x * msm}\selectfont ползволяет задать размер шрифта и расстоение между строками. msm - межстрочный множитель
    % \cyrillicfont позволяет использовать TNR для киррилицы
    % \begin{FlushRight} <текст> \end{FlushRight} позволяет выравнять текст по правому краю

\begin{Center}  {\fontsize{14}{17.5}\selectfont \cyrillicfont 
        УЧРЕЖДЕНИЕ ОБРАЗОВАНИЯ
        \par % завершает абзац, но тут нужно, чтобы перенести остальную часть на новую сроку
        “БРЕСТСКИЙ ГОСУДАРСТВЕННЫЙ ТЕХНИЧЕСКИЙ УНИВЕРСИТЕТ”
} \end{Center}

\begin{FlushLeft} % начала глобального выравнивания по левому краю
\fontsize{12}{15} % дальнейший текст 12pt

    \begin{tabular}{p{1.8cm} p{6.2cm} p{2.0cm} p{6.0cm}} 
        % параметр p укащывает тип ячейки, а количество букв p указывает количество столбцов. Можно указать ширину ячейки p{x}. Вместо p можно использовать l, r, c. Ячейки отделяются между собой символом &.
        & & & % вставляем пустую строку средствами таблицы
        \\ % переход на ряд ниже
        \textsf{Факультет} & 
        \centering \textsf{ФЭИС} \vspace{1pt} \hline &
        % \centering - выравниваем по центру
        % \vspace{2pt} - вертикальный отступ
        % \hline - рисуем линию, которая заполнит ячейку таблицы
        \textsf{Кафедра} &
        % \textsf{ <текст> } - используем ArialNarrow 
        \centering \textsf{ИИТ} \vspace{1pt} \hline 
    \end{tabular} 

    \begin{tabular}{p{2.5cm} p{5.5cm}} 
        & \\ 
        \textsf{УТВЕРЖДАЮ} & \\
        \textsf{Зав. Кафедрой} & \vspace{1pt} \hline \\ 
        \hspace{2.8cm} &  \centering \textsf{(подпись)} 
        % \hspace{2.8cm} оно нужно чтобы заполнить ячейку первого ряда, а вообще это горизонтальный отступ.
    \end{tabular} 

    \begin{tabular}{p{1.0cm} p{5.0cm} p{2.0cm}} 
        \textsf{<<18>>} & 
        \centering \textsf{апреля} \vspace{1pt} \hline &
        \textsf{2024 г.} \\
    \end{tabular}

\end{FlushLeft} % конец глобального выравнивания по левому краю

\begin{Center} 
    \fontsize{24}{30} \textsf{ЗАДАНИЕ}
    \par 
    \fontsize{14}{17.5} \textrm{по дипломному проектированию}
\end{Center}  

\begin{FlushLeft} 
\fontsize{12}{15} % дальнейший текст 12pt

    \begin{tabular}{p{1.9cm} p{14.95cm}}
        \textsf{Студенту} & 
        \textsf{Абоимову Илье Вадимовичу} \vspace{1pt} \hline 
    \end{tabular} 

    \begin{tabular}{p{2.8cm} p{14.05cm}}
        \textsf{1. Тема проекта} & 
        \textsf{Разработка модуля прогнозирования данных временных рядов технологического} \vspace{1pt} \hline  
    \end{tabular} 

    \begin{tabular}{p{2.8cm} p{14.05cm}} 
        & \textsf{процесса пастеризационной установки} \vspace{1pt} \hline 
    \end{tabular} 

    \begin{tabular}{p{5.9cm} p{4.8cm} p{0.3cm} p{5.0cm}} 
        \textsf{(Утверждена приказом по вузу от} &
        \centering \textsf{13.03.2024} \vspace{2pt} \hline &
        \centering \textsf{№} &
        \centering \textsf{249-C \hspace{0.5cm} )} \hline
    \end{tabular} 

    \begin{tabular}{p{9.4cm} p{7.45cm}} 
        \textsf{2. Сроки сдачи студентом законченного проекта} &
        \centering \textsf{дата сдачи} \vspace{1pt} \hline
    \end{tabular} 

    \begin{tabular}{p{5.7cm} p{11.15cm}} 
        \textsf{3. Исходные данные к проекту} &
        \vspace{1pt} \hline
    \end{tabular}   

    \begin{tabular}{p{17.25cm}}  
        \hspace{0.3cm} \textsf{Особенности разработки:} \vspace{1pt} \hline \\
        \hspace{0.6cm} \textsf{- Модуль должен быть развёрнут на платформе PLCnext Technology} \vspace{1pt} \hline \\
        \hspace{0.6cm} \textsf{- Текст программы модуля должен быть написан на C++}\vspace{1pt} \hline \\
        \hspace{0.6cm} \textsf{- Программа должна быть кроссплатформенной}\vspace{1pt} \hline 
    \end{tabular}   

    \begin{tabular}{p{17.25cm}} 
        \hspace{0.3cm} \textsf{Требования к функция модуля:} \vspace{1pt} \hline \\
        \hspace{0.6cm} \textsf{- Модуль должен выполнять прогнозирование технологического процесса}\vspace{1pt} \hline \\
        \hspace{0.6cm} \textsf{- Модуль должен работать в режиме реального времени}\vspace{1pt} \hline 
    \end{tabular}   

    \begin{tabular}{p{17.25cm}}
        \textsf{4. Содержание расчетно-пояснительной записки (перечень подлежащих разработке вопросов):} \vspace{1pt} \hline \\
    \end{tabular} 
    
    \begin{tabular}{p{17.25cm}} 
        \hspace{0.3cm} \textsf{ВВЕДЕНИЕ} \vspace{1pt} \hline \\
    \end{tabular} 

    \begin{tabular}{p{17.25cm}}
        \hspace{0.3cm} \textsf{1.АНАЛИЗ ПРЕДМЕТНОЙ ОБЛАСТИ И ПОСТАНОВКА ЗАДАЧИ} \vspace{1pt} \hline \\
        \hspace{0.6cm} \textsf{1.1 Основные понятия предметной области} \vspace{1pt} \hline \\
        \hspace{0.6cm} \textsf{1.2 Результаты обследования пастеризационной установки} \vspace{1pt} \hline \\
        \hspace{0.6cm} \textsf{1.3 Введение в прогнозирвоание, нейронные сети и машинное обучение} \vspace{1pt} \hline \\
        \hspace{0.6cm} \textsf{1.4 Требования к модулю прогнозирования и постановка задачи} \vspace{1pt} \hline \\
    \end{tabular}  

    \begin{tabular}{p{17.25cm}} 
        \hspace{0.3cm} \textsf{2. АНАЛИЗ СУЩЕСТВУЮЩИХ РЕШЕНИЙ} \vspace{1pt} \hline \\
    \end{tabular} 

    \begin{tabular}{p{17.25cm}} 
        \hspace{0.3cm} \textsf{3. АНАЛИЗ И ОБРАБОТКА ДАННЫХ ПАСТЕРИЗАЦИОННОЙ УСТАНОВКИ} \vspace{1pt} \hline \\
        \hspace{0.6cm} \textsf{3.1 Анализ и визуализация исходных данных технологического процесса} \vspace{1pt} \hline \\
        \hspace{0.6cm} \textsf{3.4 Обработка данных и их подготовка к прогнозированию} \vspace{1pt} \hline \\
    \end{tabular}    

    \begin{tabular}{p{17.25cm}} 
        \hspace{0.3cm} \textsf{5. РАЗРАБОТКА И ОБУЧЕНИЕ МОДЕЛИ НЕЙРОННОЙ СЕТИ} \vspace{1pt} \hline \\
        \hspace{0.6cm} \textsf{5.1 Выбор архитектуры нейронной сети} \vspace{1pt} \hline \\
        \hspace{0.6cm} \textsf{5.2 Описание средств разработки нейронной сети} \vspace{1pt} \hline \\
        \hspace{0.6cm} \textsf{5.3 Описание разработки и обучения модели нейронной сети} \vspace{1pt} \hline \\
    \end{tabular}  

    \begin{tabular}{p{17.25cm}} 
        \hspace{0.3cm} \textsf{6. ТЕСТИРОВАНИЕ И ОЦЕНИВАНИЕ МОДЕЛИ} \vspace{1pt} \hline \\
        \hspace{0.6cm} \textsf{6.1 Обзор методов тестирования и оценивания модели} \vspace{1pt} \hline \\
        \hspace{0.6cm} \textsf{6.2 Оценивание модели нейронной сети} \vspace{1pt} \hline \\
        \hspace{0.6cm} \textsf{6.3 Тестирование модели нейронной сети} \vspace{1pt} \hline \\
    \end{tabular} 

    \begin{tabular}{p{17.25cm}} 
        \hspace{0.3cm} \textsf{7. РАЗВЁРТЫВАНИЕ МОДЕЛИ НЕЙРОННОЙ СЕТИ} \vspace{1pt} \hline \\
    \end{tabular}  

    \begin{tabular}{p{17.25cm}} 
        \textsf{ЗАКЛЮЧЕНИЕ} \vspace{2pt} \hline \\
        \textsf{СПИСОК СОКРАЩЕНИЙ} \vspace{1pt} \hline \\
        \textsf{СПИСОК ЛИТЕРАТУРЫ} \vspace{2pt} \hline 
    \end{tabular} 

    \begin{tabular}{p{17.25cm}} 
        \vspace{1pt} \hline  \\
        \textsf{\textbf{6. Консультанты по проекту (с указанием относящихся к ним разделов проекта)}} \vspace{1pt} \hline \\
    \end{tabular}   

    \begin{tabular}{p{17.25cm}} 
        \vspace{1pt} \hline  \\
        \textsf{\textbf{7. Дата выдачи задания: 18.04.2024 г.}} \vspace{1pt} \hline \\
    \end{tabular} 

    \begin{tabular}{p{4.2cm} p{3.8cm} p{6.0cm} p{2.0cm}} 
        & & & 
        \\ 
        & & & 
        \\
        & \fontsize{14}{17.5} \textrm{Руководитель} & 
        \vspace{1pt} \hline & 
        \\ 
        & & \centering \fontsize{12}{15} \textsf{(подпись)} & 
        \\ 
        & & &
    \end{tabular} 

\fontsize{12}{15}

    \begin{tabular}{p{7.5cm} p{0.5cm} p{6.0cm} p{2.0cm}} % создаём таблицу
        \textsf{Задание принял к исполнению (дата)} & &
        \centering \textsf{18.04.2024} \vspace{1pt} \hline & 
        \\ % переход на ряд ниже
        & & & % вставляем пустую строку средствами таблицы
    \end{tabular} % конец таблицы

    \begin{tabular}{p{4.0cm} p{0.2cm} p{9.8cm} p{2.0cm}} % создаём таблицу
        \textsf{(подпись студента)} & &
        \vspace{1pt} \hline & 
        \\ % переход на ряд ниже
    \end{tabular} % конец таблицы

\end{FlushLeft} % конец выравнивания по левому краю

\end{document} % конец тела документа, его основной части
