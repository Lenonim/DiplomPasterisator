\documentclass[12pt, А4]{article} % задаём тип документа: статья - и его характеристики: 12-ый стандартный шрифт, размер листа А4

    % Внимание! Чтобы код ниже работал, необхоидмо переключиться на движок XeLaTe
    \usepackage{fontspec} % подключаем модуль для работы со шрифтами
    \PassOptionsToPackage{no-math}{fontspec} %  https://tex.stackexchange.com/a/26295/104425
    \usepackage{polyglossia} % Поддержка многоязычности
    \usepackage{ragged2e} % патек для нормального выравнивания текста
    \usepackage[ % задаём макет документа
        left=1.3cm, % отступ от левого края 
        right=2.8cm, % отступ от правого края
        top=2.0cm, % отсутп сверху
        bottom=1.59cm, % отступ снизу
        paperheight=11.69in, % высота листа
        paperwidth=8.5in% % ширина листа
    ]{geometry} 

    % настройка шрифтов и языков
    \defaultfontfeatures{Ligatures=TeX,Mapping=tex-text}

    \setmainlanguage[babelshorthands = true]{russian} % устанавливаемя русский как основной язык
    \setotherlanguage{english} % устанавливаем английский как дополнительный язык

    \setmainfont{Times New Roman} % устанавливаем главный шрифт 
    \setsansfont[Ligatures=TeX]{Arial} % устонавливаем дополнительный шрмфт 

    \newfontfamily\cyrillicfont[Script=Cyrillic]{Times New Roman} % делаем так, чтобы русский язык мог отображаться главным шрифтом 
    \newfontfamily\cyrillicfontsf[Script=Cyrillic]{Arial} % делаем так, чтобы русский язык мог отображаться дополнительным шрифтом 

    \newfontfamily\englishfont{Times New Roman} % делаем так, чтобы английский язык мог отображаться главным шрифтом 
    \newfontfamily\englishfontsf{Arial} % делаем так, чтобы английский язык мог отображаться дополнительным шрифтом 

    \thispagestyle{empty} % убирает нумерацию

\begin{document} % начала тела документа, его основной части


    \begin{FlushRight} % начало выравнивания по правому краю
    \fontsize{11}{13.75} \textsf{Форма №17} 
    % \textsf {} делает шрифт текста внутри скобок дополнительным шрифтом
    \end{FlushRight} % конец выравнивания по правому краю

\fontsize{14}{17.5} % дальнейший текст 14pt
% \fontsize{x}{x*1.25} контролирует размер шрифта


    \begin{Center} % начало выравниваня по центру

    \textrm{ УЧРЕЖДЕНИЕ ОБРАЗОВАНИЯ}
    % \textsf {} делает шрифт текста внутри скобок основным шрифтом
    \par % завершает абзац 
    \textrm{“БРЕСТСКИЙ ГОСУДАРСТВЕННЫЙ ТЕХНИЧЕСКИЙ УНИВЕРСИТЕТ”}

    \end{Center}  % конец выравниваня по центру

\fontsize{12}{15} % дальнейший текст 12pt

    \begin{FlushLeft} % начало выравнивания по левому краю

    \begin{tabular}{p{2.2cm} p{5.8cm} p{2.2cm} p{5.75cm}} % создаём таблицу
        % параметр p укащывает тип ячейки, а количество букв p указывает количество столбцов. Можно указать ширину ячейки p{x}. Вместо p можно использовать l, r, c.
        & & % вставляем пустую строку средствами таблицы
        \\ % переход на ряд ниже
        \textsf{Факультет} & 
        % \raggedright - выравниваем по левому краю
        \centering \textsf{ФЭИС} \vspace{1pt} \hline &
        % \centering - выравниваем по центру
        % \vspace{2pt} - вертикальный отступ
        % \hline - рисуем линию, которая заполнит ячейку таблицы
        \textsf{Кафедра} &
        \centering \textsf{ИИТ} \vspace{1pt} \hline 
        \\ % переход на ряд ниже
        & & % вставляем пустую строку средствами таблицы
    \end{tabular} % конец таблицы

    \begin{tabular}{p{2.8cm} p{5.25cm}} % создаём таблицу
        \textsf{УТВЕРЖДАЮ} & 
        \\ % переход на ряд ниже
        \textsf{Зав. Кафедры} &
        \vspace{1pt} \hline 
        \\ % переход на ряд ниже
        \hspace{2.8cm} &  \centering \textsf{(подпись)}
        % \hspace{2.8cm} оно нужно чтобы заполнить ячейку первого ряда, а вообще это горизонтальный отступ.
    \end{tabular} % конец таблицы

    \end{FlushLeft} % конец выравнивания по левому краю

    \begin{Center} % начало выравниваня по центру

        \fontsize{24}{30} \textsf{ЗАДАНИЕ}
        \par % завершает абзац 
        \fontsize{14}{17.5} \textrm{по дипломному проектирование}

    \end{Center}  % конец выравниваня по центру

\fontsize{12}{15} % дальнейший текст 12pt

    \begin{FlushLeft} % начало выравнивания по левому краю

    \begin{tabular}{p{2.2cm} p{14.65cm}} % создаём таблицу
        \textsf{Студенту} & 
        \textsf{Абоимову Илье Вадимовичу} \vspace{1pt} \hline 
    \end{tabular} % конец таблицы

    \begin{tabular}{p{3.2cm} p{13.65cm}} % создаём таблицу
        \textsf{1. Тема проекта} & 
        \textsf{Разработка модуля прогнозирования данных временных рядов} \vspace{1pt} \hline  
    \end{tabular} % конец таблицы

    \begin{tabular}{p{3.2cm} p{13.65cm}} % создаём таблицу 
        &
        \textsf{технологического процесса пастеризационной установки} \vspace{1pt} \hline 
    \end{tabular} % конец таблицы

    \begin{tabular}{p{7.0cm} p{3.8cm} p{0.5cm} p{4.7cm}} % создаём таблицу 
        \textsf{(Утверждена приказом по вузу от} &
        \centering \textsf{число} \vspace{1pt} \hline &
        \centering \textsf{№} &
        \centering \textsf{номер приказа)} \vspace{1pt} \hline
    \end{tabular} % конец таблицы

    \begin{tabular}{p{10.0cm} p{6.85cm}} % создаём таблицу 
        \textsf{2. Сроки сдачи студентом законченного проекта} &
        \centering \textsf{дата сдачи} \vspace{1pt} \hline
    \end{tabular} % конец таблицы

    \begin{tabular}{p{6.9cm} p{9.95cm}} % создаём таблицу 
        \textsf{3. Исходные данные к проекту} &
        \textsf{да} \vspace{1pt} \hline
    \end{tabular} % конец таблицы   

    \begin{tabular}{p{17.25cm}} % создаём таблицу 
        \textsf{\textbf{Общие требования к подсистеме:}} \vspace{1pt} \hline \\
        \vspace{1pt} \hline \\
        \vspace{1pt} \hline \\
        \vspace{1pt} \hline \\
        \vspace{1pt} \hline \\
        \vspace{1pt} \hline \\
        \vspace{1pt} \hline \\
        \vspace{1pt} \hline
    \end{tabular} % конец таблицы   

    \begin{tabular}{p{17.25cm}} % создаём таблицу 
        \textsf{\textbf{Требования к функция подсистемы:}} \vspace{1pt} \hline \\
        \vspace{1pt} \hline \\
        \vspace{1pt} \hline \\
        \vspace{1pt} \hline \\
        \vspace{1pt} \hline \\
        \vspace{1pt} \hline \\
        \vspace{1pt} \hline 
    \end{tabular} % конец таблицы   

    \begin{tabular}{p{17.25cm}} % создаём таблицу 
        \textsf{\textbf{4. Содержание расчетно-пояснительной записки (перечень подлежащих разра-}} \vspace{1pt} \hline \\
        \textsf{\textbf{ботке вопросов:)}} \vspace{1pt} \hline \\
    \end{tabular} % конец таблицы 
    
    \begin{tabular}{p{17.25cm}} % создаём таблицу 
        \textsf{1.ВВЕДЕНИЕ} \vspace{1pt} \hline \\
    \end{tabular} % конец таблицы 

    \begin{tabular}{p{17.25cm}} % создаём таблицу
        \textsf{2.АНАЛИЗ ПРЕДМЕТНОЙ ОБЛАСТИ И ПОСТАНОВКА ЗАДАЧИ} \vspace{1pt} \hline \\
        \hspace{0.5cm}\textsf{2.1 Основные понятия предметной области} \vspace{1pt} \hline \\
        \hspace{0.5cm}\textsf{2.2 Результаты обследования пастеризационной установки} \vspace{1pt} \hline \\
        \hspace{0.5cm}\textsf{2.3 Введение в прогнозирование и анализ временных рядов} \vspace{1pt} \hline \\
        \hspace{0.5cm}\textsf{2.4 Введение в нейро сети и машинное обучение} \vspace{1pt} \hline \\
        \hspace{0.5cm}\textsf{2.5 Требования к модулю прогназирования} \vspace{1pt} \hline \\
        \hspace{0.5cm}\textsf{2.6 Постановка задачи на создание проекта машинного обучения} \vspace{1pt} \hline \\
    \end{tabular} % конец таблицы  

    \begin{tabular}{p{17.25cm}} % создаём таблицу
        \textsf{3. АНАЛИЗ СУЩЕСТВУЮЩИХ РЕШЕНИЙ} \vspace{1pt} \hline \\
    \end{tabular} % конец таблицы

    \begin{tabular}{p{17.25cm}} % создаём таблицу
        \textsf{4. АНАЛИЗ И ОБРАБОТКА ДАННЫХ ПАСТЕРИЗАЦИОННОЙ УСТАНОВКИ} \vspace{1pt} \hline \\
        \hspace{0.5cm}\textsf{4.1 Описание исходных данных технологического процесса} \vspace{1pt} \hline \\
        \hspace{0.5cm}\textsf{4.2 Обработка и подготовка данных для прогнозирования} \vspace{1pt} \hline \\
    \end{tabular} % конец таблицы   

    \begin{tabular}{p{17.25cm}} % создаём таблицу
        \textsf{5. РАЗРАБОТКА МОДЕЛИ НЕЙРОННОЙ СЕТИ} \vspace{1pt} \hline \\
        \hspace{0.5cm}\textsf{5.1 Выбор архитектуры нейронной сети} \vspace{1pt} \hline \\
        \hspace{0.5cm}\textsf{5.2 Разработка модели нейронной сети} \vspace{1pt} \hline \\
        \hspace{0.5cm}\textsf{5.3 Обучение модели нейронной сети} \vspace{1pt} \hline \\
    \end{tabular} % конец таблицы   

    \begin{tabular}{p{17.25cm}} % создаём таблицу
        \textsf{6. ТЕСТИРОВАНИЕ И ОЦЕНИВАНИЕ МОДЕЛИ} \vspace{1pt} \hline \\
        \hspace{0.5cm}\textsf{6.1 Обзор методов тестирования и оценивания модели} \vspace{1pt} \hline \\
        \hspace{0.5cm}\textsf{6.2 Оценивание модели нейронной сети} \vspace{1pt} \hline \\
        \hspace{0.5cm}\textsf{6.3 Тестирование модели нейронной сети} \vspace{1pt} \hline \\
    \end{tabular} % конец таблицы   

    \begin{tabular}{p{17.25cm}} % создаём таблицу
        \textsf{7. РАЗВЁРТЫВАНИЕ МОДЕЛИ НЕЙРОННОЙ СЕТИ} \vspace{1pt} \hline \\
    \end{tabular} % конец таблицы  

    \begin{tabular}{p{17.25cm}} % создаём таблицу 
        \textsf{ЗАКЛЮЧЕНИЕ} \vspace{1pt} \hline \\
        \textsf{СПИСОК СОКРАЩЕНИЙ} \vspace{1pt} \hline \\
        \textsf{СПИСОК ЛИТЕРАТУРЫ} \vspace{1pt} \hline 
    \end{tabular} % конец таблицы 

    \begin{tabular}{p{17.25cm}} % создаём таблицу
        \vspace{1pt} \hline  \\
        \textsf{\textbf{6. Консультанты по проекту (с указанием относящихся к ним разделов проекта)}} \vspace{1pt} \hline \\
    \end{tabular} % конец таблицы  

    \begin{tabular}{p{17.25cm}} % создаём таблицу
        \vspace{1pt} \hline  \\
        \textsf{\textbf{7. Дата выдачи задания: дата}} \vspace{1pt} \hline \\
    \end{tabular} % конец таблицы  

    \begin{tabular}{p{4.2cm} p{3.8cm} p{6.0cm} p{2.0cm}} % создаём таблицу
        & & & 
        \\ % переход на ряд ниже
        & & & 
        \\ % переход на ряд ниже
        & \fontsize{14}{17.5} \textrm{Руководитель} & 
        \vspace{1pt} \hline & 
        \\ % переход на ряд ниже
        & & \centering \fontsize{12}{15} \textsf{(подпись)} & 
        \\ % переход на ряд ниже
        & & &
    \end{tabular} % конец таблицы

\fontsize{12}{15}

    \begin{tabular}{p{7.5cm} p{0.5cm} p{6.0cm} p{2.0cm}} % создаём таблицу
        \textsf{Задание принял к исполнению (дата)} & &
        \centering \textsf{дата} \vspace{1pt} \hline & 
        \\ % переход на ряд ниже
        & & & % вставляем пустую строку средствами таблицы
    \end{tabular} % конец таблицы

    \begin{tabular}{p{4.0cm} p{0.2cm} p{9.8cm} p{2.0cm}} % создаём таблицу
        \textsf{(подпись студента)} & &
        \vspace{1pt} \hline & 
        \\ % переход на ряд ниже
    \end{tabular} % конец таблицы

\end{FlushLeft} % конец выравнивания по левому краю

\end{document} % конец тела документа, его основной части
