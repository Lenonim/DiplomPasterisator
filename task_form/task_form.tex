\documentclass[12pt, А4, twoside]{article}

    \usepackage{fontspec}
    \PassOptionsToPackage{no-math}{fontspec}
    \usepackage{polyglossia}
    \usepackage{ragged2e}
    \usepackage[
        left=2.8cm, 
        right=1.3cm,
        top=0.7cm,
        bottom=1.59cm,
    ]{geometry}

    \defaultfontfeatures{Ligatures=TeX,Mapping=tex-text}

    \setmainlanguage[babelshorthands = true]{russian}
    \setotherlanguage{english}

    \defaultfontfeatures[ArrialNarrow]{
        Path = fonts/,
        Extension = .ttf,
        UprightFont = arialnarrow,
        BoldFont = arialnarrow_bold,
        ItalicFont = arialnarrow_italic,
        BoldItalicFont = arialnarrow_bolditalic,
    }

    \setmainfont{Times New Roman}
    \setsansfont[Ligatures=TeX]{ArialNarrow}

    \newfontfamily\cyrillicfont[Script=Cyrillic]{Times New Roman}
    \newfontfamily\cyrillicfontsf[Script=Cyrillic]{ArialNarrow}

    \newfontfamily\englishfont{Times New Roman}
    \newfontfamily\englishfontsf{ArialNarrow}

    \pagenumbering{gobble}

\newcommand\vr{\vspace{-0.1cm}}

\begin{document}

\begin{FlushRight} {\fontsize{11}{0}\selectfont \cyrillicfont
        Форма №17 \hspace{0.4 cm}
} \end{FlushRight}

\begin{Center} {\fontsize{14}{0}\selectfont \cyrillicfont
        УЧРЕЖДЕНИЕ ОБРАЗОВАНИЯ
        \par
        “БРЕСТСКИЙ ГОСУДАРСТВЕННЫЙ ТЕХНИЧЕСКИЙ УНИВЕРСИТЕТ”
} \end{Center}

\begin{FlushLeft}
\fontsize{12}{15}

    \begin{tabular}{p{1.8cm} p{6.2cm} p{2.0cm} p{6.0cm}}
        \textsf{Факультет} &
        \centering \textsf{ФЭИС} \vspace{0pt} \hline &
        \textsf{Кафедра} &
        \centering \textsf{ИИТ} \vspace{0pt} \hline
    \end{tabular}

    \begin{tabular}{p{2.5cm} p{5.5cm}}
        & \\
        \textsf{УТВЕРЖДАЮ} & \\
        \textsf{Зав. Кафедрой} & \vspace{0pt} \hline \\
        \hspace{2.8cm} &  \centering \textsf{(подпись)}
    \end{tabular}

\vspace{-0.4 cm}

    \begin{tabular}{p{1.0cm} p{5.0cm} p{2.0cm}}
        \textsf{<< \hspace{4mm} >>} &
        \vspace{0pt} \hline &
        \textsf{2024 г.} \\
    \end{tabular}

\end{FlushLeft}

\begin{Center}
    \fontsize{24}{30} \textsf{ЗАДАНИЕ}
    \par 
    \fontsize{14}{17.5} \textrm{по дипломному проектированию}
\end{Center}

\begin{FlushLeft}
\fontsize{12}{0}

    \begin{tabular}{p{1.9cm} p{14.95cm}}
        \textsf{Студенту} &
        \textsf{Абоимову Илье Вадимовичу} \vspace{0pt} \hline
    \end{tabular}

    \vspace{-0.1 cm}

    \begin{tabular}{p{2.8cm} p{14.05cm}}
        \textsf{1. Тема проекта} &
        \textsf{Разработка модуля прогнозирования данных временных рядов технологического} \vspace{0pt} \hline
    \end{tabular}

    \vspace{-0.1 cm}

    \begin{tabular}{p{2.8cm} p{14.05cm}}
        & \textsf{процесса пастеризационной установки} \vspace{0pt} \hline
    \end{tabular}

    \vspace{-0.1 cm}

    \begin{tabular}{p{5.9cm} p{4.8cm} p{0.3cm} p{5.0cm}}
        \textsf{(Утверждена приказом по вузу от} &
        \centering \textsf{13.03.2024} \vspace{1pt} \hline &
        \centering \textsf{№} &
        \centering \textsf{249-C \hspace{0.5cm} )} \hline
    \end{tabular}

    \vspace{-0.1 cm}

    \begin{tabular}{p{9.4cm} p{7.45cm}}
        \textsf{2. Сроки сдачи студентом законченного проекта} &
        \centering \textsf{07.06.2024} \vspace{1pt} \hline
    \end{tabular}

    \vspace{-0.1 cm}

    \begin{tabular}{p{5.7cm} p{11.15cm}}
        \textsf{3. Исходные данные к проекту} &
        \vspace{1pt} \hline
    \end{tabular}

    \begin{tabular}{p{17.25cm}}
        \vspace{0pt} \hline \\
    \end{tabular}

    \vspace{-0.1 cm}

    \begin{tabular}{p{17.25cm}}
        \hspace{0.3cm} \textsf{Особенности разработки:} \vspace{0pt} \hline \\
    \end{tabular}

    \vspace{-0.1 cm}

    \begin{tabular}{p{17.25cm}}
        \hspace{0.6cm} \textsf{- Модуль должен быть развёрнут на платформе PLCnext Technology} \vspace{0pt} \hline \\
    \end{tabular}

    \vspace{-0.1 cm}

    \begin{tabular}{p{17.25cm}}
        \hspace{0.6cm} \textsf{- Текст программы модуля должен быть написан на C++}\vspace{0pt} \hline \\
    \end{tabular}

    \vspace{-0.1 cm}

    \begin{tabular}{p{17.25cm}}
        \hspace{0.6cm} \textsf{- Программа должна быть кроссплатформенной}\vspace{0pt} \hline 
    \end{tabular}

    \begin{tabular}{p{17.25cm}}
        \vspace{0pt} \hline \\
    \end{tabular}

    \vspace{-0.1 cm}

    \begin{tabular}{p{17.25cm}}
        \hspace{0.3cm} \textsf{Требования к функция модуля:} \vspace{0pt} \hline \\
    \end{tabular}

    \vspace{-0.1 cm}

    \begin{tabular}{p{17.25cm}}
        \hspace{0.6cm} \textsf{- Модуль должен выполнять прогнозирование технологического процесса}\vspace{0pt} \hline \\
    \end{tabular}

    \vspace{-0.1 cm}

    \begin{tabular}{p{17.25cm}}
        \hspace{0.6cm} \textsf{- Модуль должен работать в режиме реального времени}\vspace{0pt} \hline 
    \end{tabular}

    \vspace{-0.1 cm}

    \begin{tabular}{p{17.25cm}}
        \hspace{0.6cm} \textsf{- Модуль должен учитывать зависимости между параметрами технологического процесса}\vspace{0pt} \hline 
    \end{tabular}

    \vspace{-0.1 cm}

    \begin{tabular}{p{17.25cm}}
        \vspace{0.1cm} \hline \\
        \textsf{4. Содержание расчетно-пояснительной записки (перечень подлежащих разработке вопросов):} \vspace{0pt} \hline \\
    \end{tabular}

    \begin{tabular}{p{17.25cm}}
        \vspace{0pt} \hline \\
    \end{tabular}

    \vspace{-0.1 cm}

    \begin{tabular}{p{17.25cm}}
        \hspace{0.3cm} \textsf{ВВЕДЕНИЕ} \vspace{0pt} \hline \\
    \end{tabular}

    \vspace{-0.1 cm}

    \begin{tabular}{p{17.25cm}}
        \hspace{0.3cm} \textsf{1. АНАЛИЗ ПРЕДМЕТНОЙ ОБЛАСТИ И ПОСТАНОВКА ЗАДАЧИ} \vspace{0pt} \hline \\
    \end{tabular}

    \vspace{-0.1 cm}

    \begin{tabular}{p{17.25cm}}
        \hspace{0.3cm} \textsf{2. АНАЛИЗ СУЩЕСТВУЮЩИХ РЕШЕНИЙ} \vspace{0pt} \hline \\
    \end{tabular}

    \vspace{-0.1 cm}

    \begin{tabular}{p{17.25cm}}
        \hspace{0.3cm} \textsf{3. АНАЛИЗ И ОБРАБОТКА ДАННЫХ ПАСТЕРИЗАЦИОННОЙ УСТАНОВКИ} \vspace{0pt} \hline \\
    \end{tabular}

    \vspace{-0.1 cm}

    \begin{tabular}{p{17.25cm}}
        \hspace{0.3cm} \textsf{4. РАЗРАБОТКА, ОБУЧЕНИЕ И ОЦЕНКА МОДЕЛИ НЕЙРОННОЙ СЕТИ} \vspace{0pt} \hline \\
    \end{tabular}

    \vspace{-0.1 cm}

    \begin{tabular}{p{17.25cm}}
        \hspace{0.3cm} \textsf{5. РАЗВЁРТЫВАНИЕ МОДЕЛИ НЕЙРОННОЙ СЕТИ} \vspace{0pt} \hline \\
    \end{tabular}

    \vspace{-0.1 cm}

    \begin{tabular}{p{17.25cm}} 
        \hspace{0.3cm} \textsf{6. ТЕХНИКО-ЭКОНОМИЧЕСКОЕ ОБОСНОВАНИЕ} \vspace{1pt} \hline \\
    \end{tabular}  

    \vspace{-0.1 cm}

    \begin{tabular}{p{17.25cm}}
        \hspace{0.3cm} \textsf{ЗАКЛЮЧЕНИЕ} \vspace{1pt} \hline \\
    \end{tabular}

    \vspace{-0.1 cm}

    \begin{tabular}{p{17.25cm}}
        \hspace{0.3cm} \textsf{СПИСОК СОКРАЩЕНИЙ} \vspace{0pt} \hline \\
    \end{tabular}

    \vspace{-0.1 cm}

    \begin{tabular}{p{17.25cm}}
        \hspace{0.3cm} \textsf{СПИСОК ЛИТЕРАТУРЫ} \vspace{1pt} \hline  \\
    \end{tabular}

    \vspace{-0.1 cm}

    \begin{tabular}{p{17.25cm}}
        \hspace{0.3cm} \textsf{ПРИЛОЖЕНИЕ А: ТЕКСТ ПРОГРАММЫ} \vspace{1pt} \hline  \\
    \end{tabular}

    \begin{tabular}{p{17.25cm}}
        \vspace{0pt} \hline \\
    \end{tabular}

    \begin{tabular}{p{17.25cm}}
        \vspace{0pt} \hline \\
    \end{tabular}

    \vspace{0.1 cm}

    \begin{tabular}{p{17.25cm}}
        \textsf{5. Перечень графического материала (с точным указанием обязательных чертежей и графиков)} \vspace{0pt} \hline \\
    \end{tabular}

    \begin{tabular}{p{17.25cm}}
        \vspace{0pt} \hline \\
    \end{tabular}

    \vspace{-0.1 cm}

    \begin{tabular}{p{17.25cm}}
        \hspace{0.3cm} \textsf{1. Постановка задачи (плакат {--} формат А1)} \vspace{0pt} \hline  \\
    \end{tabular}

    \vspace{-0.1 cm}

    \begin{tabular}{p{17.25cm}}
        \hspace{0.3cm} \textsf{2. Схема пастеризационной установки (плакат {--} формат А1)} \vspace{0pt} \hline  \\
    \end{tabular}

    \vspace{-0.1 cm}

    \begin{tabular}{p{17.25cm}}
        \hspace{0.3cm} \textsf{3. Структура ячейки классической LSTM сети (плакат {--} формат А1)} \vspace{0pt} \hline  \\
    \end{tabular}

    \vspace{-0.1 cm}

    \begin{tabular}{p{17.25cm}}
        \hspace{0.3cm} \textsf{4. Схема взаимодействия программ (чертёж {--} формат А1)} \vspace{0pt} \hline  \\
    \end{tabular}

    \vspace{-0.1 cm}

    \begin{tabular}{p{17.25cm}}
        \hspace{0.3cm} \textsf{5. Блок-схема алгоритма прогнозирования (чертёж {--} формат А1)} \vspace{0pt} \hline  \\
    \end{tabular}

    \begin{tabular}{p{17.25cm}}
        \vspace{0pt} \hline  \\
    \end{tabular}

    \vspace{-0.1 cm}

    \begin{tabular}{p{17.25cm}}
        \textsf{6. Консультанты по проекту (с указанием относящихся к ним разделов проекта)} \vspace{0pt} \hline \\
    \end{tabular}

    \begin{tabular}{p{17.25cm}}
        \vspace{0pt} \hline \\
    \end{tabular}

    \vspace{-0.1 cm}

    \begin{tabular}{p{17.25cm}}
        \hspace{0.3cm} \textsf{ст. преподаватель Иванюк Д. С.} \hspace{4.065cm} \textsf{{--} по основным разделам} \vspace{0pt} \hline \\
    \end{tabular}

    \vspace{-0.1 cm}

    \begin{tabular}{p{17.25cm}}
        \hspace{0.3cm} \textsf{ассистент кафедры ИИТ Булей Е. В.} \hspace{3.3325cm} \textsf{{--} нормоконтроль} \vspace{0pt} \hline \\
    \end{tabular}

    \vspace{-0.1 cm}

    \begin{tabular}{p{17.25cm}}
        \hspace{0.3cm} \textsf{доцент кафедры МЭМИ, к.э.н., доцент Четырбок Н. П.} \hspace{0.5cm} \textsf{{--} экономический раздел}\vspace{0pt} \hline \\
    \end{tabular}

    \begin{tabular}{p{17.25cm}}
        \vspace{0pt} \hline \\
    \end{tabular}

    \vspace{-0.1 cm}

    \begin{tabular}{p{17.25cm}}
        \textsf{7. Дата выдачи задания: 19.04.2024 г.} \vspace{0pt} \hline \\
    \end{tabular}

    \begin{tabular}{p{17.25cm}}
        \vspace{0pt} \hline \\
    \end{tabular}

    \vspace{-0.1 cm}

    \begin{tabular}{p{17.25cm}}
        \textsf{8. Календарный график работы над проектом на весь период проектирования (с указанием сроков} \vspace{0pt} \hline \\
    \end{tabular}

    \vspace{-0.1 cm}

    \begin{tabular}{p{17.25cm}}
        \textsf{выполнения и трудоемкость отдельных этапов)} \vspace{0pt} \hline \\
    \end{tabular}

    \begin{tabular}{p{17.25cm}}
        \vspace{0pt} \hline \\
    \end{tabular}

    \vspace{-0.1 cm}

    \begin{tabular}{p{17.25cm}}
        \hspace{0.3cm} \textsf{Раздел 1:} \hspace{2.54cm} \textsf{ 19.04 {--} 25.04:} \hspace{0.5cm} \textsf{10\%} \vspace{0pt} \hline \\
    \end{tabular}

    \vspace{-0.1 cm}

    \begin{tabular}{p{17.25cm}}
        \hspace{0.3cm} \textsf{Раздел 2:} \hspace{2.54cm} \textsf{ 26.04 {--} 29.04:} \hspace{0.5cm} \textsf{5\%} \vspace{0pt} \hline \\
    \end{tabular}

    \vspace{-0.1 cm}

    \begin{tabular}{p{17.25cm}}
        \hspace{0.3cm} \textsf{Раздел 3:} \hspace{2.54cm} \textsf{ 30.04 {--} 05.05:} \hspace{0.5cm} \textsf{20\%} \vspace{0pt} \hline \\
    \end{tabular}

    \vspace{-0.1 cm}

    \begin{tabular}{p{17.25cm}}
        \hspace{0.3cm} \textsf{Раздел 4:} \hspace{2.54cm} \textsf{ 06.05 {--} 12.05:} \hspace{0.5cm} \textsf{20\%} \vspace{0pt} \hline \\
    \end{tabular}

    \vspace{-0.1 cm}

    \begin{tabular}{p{17.25cm}}
        \hspace{0.3cm} \textsf{Раздел 5:} \hspace{2.54cm} \textsf{ 13.05 {--} 18.05:} \hspace{0.5cm} \textsf{10\%} \vspace{0pt} \hline \\
    \end{tabular}

    \vspace{-0.1 cm}

    \begin{tabular}{p{17.25cm}}
        \hspace{0.3cm} \textsf{Раздел 6:} \hspace{2.54cm} \textsf{ 19.05 {--} 25.05:} \hspace{0.5cm} \textsf{15\%} \vspace{0pt} \hline \\
    \end{tabular}

    \vspace{-0.1 cm}

    \begin{tabular}{p{17.25cm}}
        \hspace{0.3cm} \textsf{Оформление проекта:} \hspace{0.5cm} \textsf{ 02.06 {--} 07.06:} \hspace{0.5cm} \textsf{20\%} \vspace{0pt} \hline \\
    \end{tabular}

    \begin{tabular}{p{17.25cm}}
        \vspace{0pt} \hline \\
        \vspace{0pt} \hline \\
        \vspace{0pt} \hline \\
        \vspace{0pt} \hline \\
        \vspace{0pt} \hline \\
    \end{tabular}

    \begin{tabular}{p{4.2cm} p{3.8cm} p{6.0cm} p{2.0cm}}
        & & &
        \\
        & & &
        \\
        & \fontsize{14}{17.5} \textsf{Руководитель} &
        \vspace{0pt} \hline &
        \\
        & & \centering \fontsize{12}{15} \textsf{(подпись)} &
        \\
        & & &
    \end{tabular}

\fontsize{12}{15}

    \begin{tabular}{p{7.5cm} p{0.5cm} p{6.0cm} p{2.0cm}}
        \textsf{Задание принял к исполнению (дата)} & &
        \centering \textsf{19.04.2024} \vspace{1pt} \hline &
        \\
        & & &
    \end{tabular}

    \begin{tabular}{p{4.0cm} p{0.2cm} p{9.8cm} p{2.0cm}}
        \textsf{(подпись студента)} & &
        \vspace{0pt} \hline &
        \\ 
    \end{tabular}

\end{FlushLeft}

\end{document}
