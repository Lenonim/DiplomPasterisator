\documentclass[12pt, А4, twoside]{article} 

    \usepackage{fontspec} 
    \PassOptionsToPackage{no-math}{fontspec} 
    \usepackage{polyglossia}
    \usepackage{ragged2e} 
    \usepackage[ 
        left=2.8cm,  
        right=1.3cm, 
        top=2.0cm, 
        bottom=1.59cm,
    ]{geometry} 

    \defaultfontfeatures{Ligatures=TeX,Mapping=tex-text}

    \setmainlanguage[babelshorthands = true]{russian} 
    \setotherlanguage{english}

    \defaultfontfeatures[ArrialNarrow]{
        Path = fonts/,
        Extension = .ttf,
        UprightFont = arialnarrow,
        BoldFont = arialnarrow_bold,
        ItalicFont = arialnarrow_italic,
        BoldItalicFont = arialnarrow_bolditalic,
    }

    \setmainfont{Times New Roman}  
    \setsansfont[Ligatures=TeX]{ArialNarrow} 

    \newfontfamily\cyrillicfont[Script=Cyrillic]{Times New Roman}  
    \newfontfamily\cyrillicfontsf[Script=Cyrillic]{ArialNarrow}  

    \newfontfamily\englishfont{Times New Roman}  
    \newfontfamily\englishfontsf{ArialNarrow} 

    \pagenumbering{gobble} 

\begin{document} 

\begin{FlushRight}{\fontsize{11}{13.75}\selectfont \cyrillicfont
        Форма №17
} \end{FlushRight}

\begin{Center}  {\fontsize{14}{17.5}\selectfont \cyrillicfont 
        УЧРЕЖДЕНИЕ ОБРАЗОВАНИЯ
        \par 
        “БРЕСТСКИЙ ГОСУДАРСТВЕННЫЙ ТЕХНИЧЕСКИЙ УНИВЕРСИТЕТ”
} \end{Center}

\begin{FlushLeft}
\fontsize{12}{15}

    \begin{tabular}{p{1.8cm} p{6.2cm} p{2.0cm} p{6.0cm}} 
        & & & 
        \\ 
        \textsf{Факультет} & 
        \centering \textsf{ФЭИС} \vspace{1pt} \hline &
        \textsf{Кафедра} &
        \centering \textsf{ИИТ} \vspace{1pt} \hline 
    \end{tabular} 

    \begin{tabular}{p{2.5cm} p{5.5cm}} 
        & \\ 
        \textsf{УТВЕРЖДАЮ} & \\
        \textsf{Зав. Кафедрой} & \vspace{1pt} \hline \\ 
        \hspace{2.8cm} &  \centering \textsf{(подпись)} 
    \end{tabular} 

    \begin{tabular}{p{1.0cm} p{5.0cm} p{2.0cm}} 
        \textsf{<< \hspace{4mm} >>} & 
        \vspace{1pt} \hline &
        \textsf{2024 г.} \\
    \end{tabular}

\end{FlushLeft}

\begin{Center} 
    \fontsize{24}{30} \textsf{ЗАДАНИЕ}
    \par 
    \fontsize{14}{17.5} \textrm{по дипломному проектированию}
\end{Center}  

\begin{FlushLeft} 
\fontsize{12}{15} 

    \begin{tabular}{p{1.9cm} p{14.95cm}}
        \textsf{Студенту} & 
        \textsf{Абоимову Илье Вадимовичу} \vspace{1pt} \hline 
    \end{tabular} 

    \begin{tabular}{p{2.8cm} p{14.05cm}}
        \textsf{1. Тема проекта} & 
        \textsf{Разработка модуля прогнозирования данных временных рядов технологического} \vspace{1pt} \hline  
    \end{tabular} 

    \begin{tabular}{p{2.8cm} p{14.05cm}} 
        & \textsf{процесса пастеризационной установки} \vspace{1pt} \hline 
    \end{tabular} 

    \begin{tabular}{p{5.9cm} p{4.8cm} p{0.3cm} p{5.0cm}} 
        \textsf{(Утверждена приказом по вузу от} &
        \centering \textsf{13.03.2024} \vspace{2pt} \hline &
        \centering \textsf{№} &
        \centering \textsf{249-C \hspace{0.5cm} )} \hline
    \end{tabular} 

    \begin{tabular}{p{9.4cm} p{7.45cm}} 
        \textsf{2. Сроки сдачи студентом законченного проекта} &
        \centering \textsf{дата сдачи} \vspace{1pt} \hline
    \end{tabular} 

    \begin{tabular}{p{5.7cm} p{11.15cm}} 
        \textsf{3. Исходные данные к проекту} &
        \vspace{1pt} \hline
    \end{tabular}   

    \begin{tabular}{p{17.25cm}}  
        \hspace{0.3cm} \textsf{Особенности разработки:} \vspace{1pt} \hline \\
        \hspace{0.6cm} \textsf{- Модуль должен быть развёрнут на платформе PLCnext Technology} \vspace{1pt} \hline \\
        \hspace{0.6cm} \textsf{- Текст программы модуля должен быть написан на C++}\vspace{1pt} \hline \\
        \hspace{0.6cm} \textsf{- Программа должна быть кроссплатформенной}\vspace{1pt} \hline 
    \end{tabular}   

    \begin{tabular}{p{17.25cm}} 
        \hspace{0.3cm} \textsf{Требования к функция модуля:} \vspace{1pt} \hline \\
        \hspace{0.6cm} \textsf{- Модуль должен выполнять прогнозирование технологического процесса}\vspace{1pt} \hline \\
        \hspace{0.6cm} \textsf{- Модуль должен работать в режиме реального времени}\vspace{1pt} \hline 
    \end{tabular}   

    \begin{tabular}{p{17.25cm}}
        \vspace{1pt} \hline \\
        \textsf{4. Содержание расчетно-пояснительной записки (перечень подлежащих разработке вопросов):} \vspace{1pt} \hline \\
    \end{tabular} 
    
    \begin{tabular}{p{17.25cm}} 
        \hspace{0.3cm} \textsf{ВВЕДЕНИЕ} \vspace{1pt} \hline \\
    \end{tabular} 

    \begin{tabular}{p{17.25cm}}
        \hspace{0.3cm} \textsf{1. АНАЛИЗ ПРЕДМЕТНОЙ ОБЛАСТИ И ПОСТАНОВКА ЗАДАЧИ} \vspace{1pt} \hline \\
    \end{tabular}  

    \begin{tabular}{p{17.25cm}} 
        \hspace{0.3cm} \textsf{2. АНАЛИЗ СУЩЕСТВУЮЩИХ РЕШЕНИЙ} \vspace{1pt} \hline \\
    \end{tabular} 

    \begin{tabular}{p{17.25cm}} 
        \hspace{0.3cm} \textsf{3. АНАЛИЗ И ОБРАБОТКА ДАННЫХ ПАСТЕРИЗАЦИОННОЙ УСТАНОВКИ} \vspace{1pt} \hline \\
    \end{tabular}    

    \begin{tabular}{p{17.25cm}} 
        \hspace{0.3cm} \textsf{4. РАЗРАБОТКА И ОБУЧЕНИЕ МОДЕЛИ НЕЙРОННОЙ СЕТИ} \vspace{1pt} \hline \\
    \end{tabular}  

    \begin{tabular}{p{17.25cm}} 
        \hspace{0.3cm} \textsf{6. ТЕСТИРОВАНИЕ И ОЦЕНКА МОДЕЛИ} \vspace{1pt} \hline \\
    \end{tabular} 

    \begin{tabular}{p{17.25cm}} 
        \hspace{0.3cm} \textsf{7. РАЗВЁРТЫВАНИЕ МОДЕЛИ НЕЙРОННОЙ СЕТИ} \vspace{1pt} \hline \\
    \end{tabular}  

    \begin{tabular}{p{17.25cm}} 
        \hspace{0.3cm} \textsf{8. ТЕХНИКО-ЭКОНОМИЧЕСКОЕ ОБОСНОВАНИЕ} \vspace{2pt} \hline \\
    \end{tabular}  

    \begin{tabular}{p{17.25cm}} 
        \textsf{ЗАКЛЮЧЕНИЕ} \vspace{2pt} \hline \\
    \end{tabular}  

    \begin{tabular}{p{17.25cm}} 
        \textsf{СПИСОК СОКРАЩЕНИЙ} \vspace{1pt} \hline \\
    \end{tabular}  

    \begin{tabular}{p{17.25cm}} 
        \textsf{СПИСОК ЛИТЕРАТУРЫ} \vspace{2pt} \hline  \\
    \end{tabular}  

    \begin{tabular}{p{17.25cm}} 
        \vspace{1pt} \hline  \\
        \textsf{5. Перечень графического материала (с точным указанием обязательных чертежей и графиков)} \vspace{1pt} \hline \\
        \vspace{1pt} \hline  \\
    \end{tabular} 

    \begin{tabular}{p{17.25cm}} 
        \vspace{1pt} \hline  \\
        \textsf{6. Консультанты по проекту (с указанием относящихся к ним разделов проекта)} \vspace{1pt} \hline \\
        \hspace{0.3cm} \textsf{ст. преподаватель Иванюк Д. С. {--} по основным разделам} \vspace{1pt} \hline \\
        \hspace{0.3cm} \textsf{ст. преподаватель {--} нормоконтроль} \vspace{1pt} \hline \\
        \hspace{0.3cm} \textsf{доцент кафедры МЭМИ, к.э.н., доцент Четырбок Н. П. {--} экономический раздел} \vspace{1pt} \hline \\
    \end{tabular}   

    \begin{tabular}{p{17.25cm}} 
        \vspace{1pt} \hline \\
        \textsf{7. Дата выдачи задания: 19.04.2024 г.} \vspace{1pt} \hline \\
    \end{tabular} 

    \begin{tabular}{p{17.25cm}} 
        \vspace{1pt} \hline \\
        \textsf{8. Календарный график работы над проектом на весь период проектирования (с указанием} \vspace{1pt} \hline \\
        \textsf{сроков выполнения и трудоемкость отдельных этапов)} \vspace{1pt} \hline \\
        \vspace{1pt} \hline \\
    \end{tabular} 

    \begin{tabular}{p{4.2cm} p{3.8cm} p{6.0cm} p{2.0cm}} 
        & & & 
        \\ 
        & & & 
        \\
        & \fontsize{14}{17.5} \textrm{Руководитель} & 
        \vspace{1pt} \hline & 
        \\ 
        & & \centering \fontsize{12}{15} \textsf{(подпись)} & 
        \\ 
        & & &
    \end{tabular} 

\fontsize{12}{15}

    \begin{tabular}{p{7.5cm} p{0.5cm} p{6.0cm} p{2.0cm}} 
        \textsf{Задание принял к исполнению (дата)} & &
        \centering \textsf{19.04.2024} \vspace{1pt} \hline & 
        \\ 
        & & & 
    \end{tabular} 

    \begin{tabular}{p{4.0cm} p{0.2cm} p{9.8cm} p{2.0cm}}
        \textsf{(подпись студента)} & &
        \vspace{1pt} \hline & 
        \\ 
    \end{tabular} 

\end{FlushLeft} 

\end{document}
